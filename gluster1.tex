\chapter{O \textit{gluster não monta}}

Há um bug no \texttt{gluster 4.1.5} que evita a montagem de volumes.
No nosso caso, ocorre mesmo quando tenta-se montar apenas os volumes localizados na \textit{playerone}, ou seja, o texttt{gv-apps}.

Caso o comando \texttt{mount-a} ou \texttt{mount -t glusterfs playerone:/gv-apps /gv/apps} não funcionem (exemplo abaixo):

\begin{lstlisting}[language=bash,basicstyle=\small]
  Erro. Olhar log.
\end{lstlisting}

Uma solução encontrada é parar o volume:

\begin{lstlisting}[language=bash,basicstyle=\small]
[root@playerone ~]# gluster volume stop gv-apps
Stopping volume will make its data inaccessible. Do you want to continue? (y/n) y
volume stop: gv-apps: success
\end{lstlisting}

Logo após é necessário iniciar o volume:

\begin{lstlisting}[language=bash,basicstyle=\small]
[root@playerone ~]# gluster volume start gv-apps
volume stop: gv-apps: success
\end{lstlisting}

É possível verificar se o volume subiu corretamente com o seguinte comando:

\begin{lstlisting}[language=bash,basicstyle=\small]
[root@playerone ~]# gluster volume status
Status of volume: gv-apps
Gluster process                             TCP Port  RDMA Port  Online  Pid
------------------------------------------------------------------------------
Brick playerone:/brick/p0/apps              49155     0          Y       22652
Brick playerone:/brick/p1/apps              49156     0          Y       22674
Brick playerone:/brick/p2/apps              49157     0          Y       22696
 
Task Status of Volume gv-apps
------------------------------------------------------------------------------
There are no active volume tasks
 
Volume gv-data is not started

\end{lstlisting}

Quando os volumes estão corretamente iniciados, as portas IP aparecem numeradas,
uma porta para cada \textit{brick}.

Este procedimento feito, ainda é necessário montar os volumes no sistema de arquivos. Isso pode ser feito com os comandos \texttt{mount -a} que vai acessar a tabela de montagem (\texttt{/etc/fstab}) ou com o comando específico para montagem dos volumes gluster: \texttt{mount -t glusterfs playerone:/gv-apps /gv/app}.

Se o volume foi corretamente montado, será possível vê-lo com o comando \texttt{df -h | grep gv}:

\begin{lstlisting}[language=bash,basicstyle=\small]
[root@playerone ~]# df -h | grep gv
playerone:/gv-apps                1,3T  198G  1,1T  16% /gv/apps
\end{lstlisting}




