%Tudo que começa com '%' é comentário e é ignorado pelo compilador

%Gerando arquivo em latex:
%latex arquivo.tex (em dvi)
%pdflatex arquivo (em pdf)
%dvipdfm arquivo
%s2pdf arquivo

% Alguns modos de usar o latex:
% Windows – Miktex com Led
% Linux – texlive com kile

\documentclass[11pt,a4]{report} %aqui fala o tipo de documento e o tamanho da fonte. Opções: tamanho do texto (10pt, 12pt, 14pt), formato do papel (a4paper, a5paper, b5paper, letterpaper, legalpaper, executivepaper), o número de colunas (onecolumn, twocolumn), entre outras opções.
%Por exemplo, [12pt,a4,twocolumn].
%classe: article, report, letter, book ou slides. Instalar abnt para quem está pensando no tf
\usepackage[brazilian]{babel} %hifenização em português do brasil
\usepackage[T1]{fontenc} % caracteres com acentos são considerados um bloco só
\usepackage[utf8]{inputenc} % Corrigie os acentos em Português
\usepackage{ae} %arruma a fonte quando usa o pacote fontenc
%\usepackage[pdftex]{graphicx}%Para inserir figuras
\usepackage{vhistory}
\usepackage{xr} % referenciar arquivos externos
\usepackage{hyperref} % Pra link http funcionar
\usepackage{graphicx} % Pra incluir o logo do laboratorio
\usepackage{xcolor}
\usepackage{newunicodechar}
\usepackage{textcomp} % Nao sei pra que serve, mas esta por causa do listings
\usepackage{courier} % Para poder usar fonte courier no listings
\usepackage{listings} % Pra poder listar como comandos bash
\lstset{upquote=true} % pras aspas ficarem retas
\lstset{basicstyle=\footnotesize\ttfamily,breaklines=true} %formata o listings em courier 
\lstset{framextopmargin=50pt,frame=bottomline}

%\externaldocument{prana}
%\externaldocument{pytera}


\begin{document}

\begin{titlepage}
    \centering
    \vfill
    {\bfseries\Large
		TO-DO para sistemas do \textit{Cluster} LTHN\\
		\vskip2cm
	    \includegraphics[width=8cm]{lthn.png} % also works with logo.pdf

        \vskip2cm
		Vitor Vasconcelos Araújo Silva\\ 
		vitors@cdtn.br\\
		\vskip3cm
		\today\\

    }    
    \vfill
\end{titlepage}


%\maketitle %cria o título

%\def \negritovi {\textbf} %Criando comandos

\tableofcontents %índice
%\pagebreak % Quebra de página
%\listoffigures %indice de figuras
%\listoftables %indice de tabelas
%\pagebreak % Quebra de página

\begin{abstract}
  Este documento contém ações imediatas relacionadas a diferentes sistemas do \textit{cluster} do CDTN.
  O objetivo dele é servir como referência rápida para recuperação do sistema em casos de falhas ou
  eventos que levem a perda da funcionalidade de algum serviço do sistema. Pretende-se que este documento
  esteja em constante evolução e vá, paulatinamente, incorporando novas ações à medida em que se apresentem
  eventos de falhas ou de configuração novos. 
\end{abstract}

%\begin{versionhistory}
%	\vhEntry{0.0}{21/11/2018}{Vitor}{Documento criado}
%	\vhEntry{0.1}{30/11/2018}{Vitor}{Correção e Atualização}
%	\vhEntry{0.2}{07/12/2018}{Vitor}{Correção e Atualização}
%	\vhEntry{0.3}{13/12/2018}{Vitor}{Correção e Atualização}
%%	\vhEntry{1.1}{23.01.04}{DP|JPW}{correction}
%%    \vhEntry{1.2}{03.02.04}{DP|JPW}{revised after review}
%\end{versionhistory}

\setcounter{chapter}{-1}

\chapter{Como usar este documento?}

\textbf{Não} há neste texto referência ao funcionamento, arquitetura
  ou topologia do \textit{cluster} LTHN. Essas informações estão em outros documentos referenciados. Tampouco pretende-se
  que seja usado por quem não tenha conhecimento avançado do sistema operacional Linux.\\
  
  \textbf{Todos} os capítulos e seções assumem que o usuário já esteja logado no sistema como superusuário (\texttt{root}).\\
  Há alguns scripts implementados com o objetivo de replicar comandos em todos os nós. O script \texttt{bsh} executa o comando dado sequencialmente em todas as máquinas nó. Uso:

  \begin{lstlisting}[language=bash,basicstyle=\small]
    $ bsh {comando arg1 arg2 ...}
  \end{lstlisting}

  \chapter{Ligando o sistema}
\label{chap:turn-on}

Em algumas ocasiões - falha no sistema de ar condicionado ou prolongada falta de energia não atendida pelo no-break - o \textit{cluster} deverá ser desligado ou desligará por si só. Como proceder:

\begin{enumerate}
    \item Religar todos os computadores e o NAS e aguardar que todos terminem o processo de boot. 
\end{enumerate}

Os volumes \textit{gluster} não serão montados automaticamente. Isso pode ser visto com o comando:

\begin{lstlisting}[language=bash,basicstyle=\small]
[root@playerone ~] df -h | grep gv
[root@playerone ~] 
\end{lstlisting}

\section{Montar volumes}
% ---------------------------------------------------------------------------------------------------------
Nada retornado mostra que os volumes não estão montados. Para montar todos os volumes do sistema (notar o comando bsh montando para todos os nós):

\begin{lstlisting}[language=bash,basicstyle=\small]
[root@playerone cfx] mount -a  
[root@playerone cfx] df -h | grep gv
playerone:/gv-apps              1.1T  191G  932G  17% /gv/apps
playerone:/gv-data              2.5T  526G  2.0T  22% /gv/data
[root@playerone cfx] 
[root@playerone cfx] bsh mount -a
*****irineu*****
*****insolente*****
*****galo*****
*****marvin*****
*****mordac*****
*****demoninho*****
*****zumbi*****
*****coragem*****
[root@playerone cfx] bsh 'df -h | grep gv'
*****irineu*****
irineu:/gv-data                 2.5T  526G  2.0T  22% /gv/data
playerone:/gv-apps              1.1T  191G  932G  17% /gv/apps
*****insolente*****
playerone:/gv-apps              1.1T  191G  932G  17% /gv/apps
insolente:/gv-data              2.5T  526G  2.0T  22% /gv/data
*****galo*****
playerone:/gv-apps              1.1T  191G  932G  17% /gv/apps
galo:/gv-data                   2.5T  526G  2.0T  22% /gv/data
*****marvin*****
marvin:/gv-data                 2.5T  526G  2.0T  22% /gv/data
playerone:/gv-apps              1.1T  191G  932G  17% /gv/apps
*****mordac*****
mordac:/gv-data                 2.5T  526G  2.0T  22% /gv/data
playerone:/gv-apps              1.1T  191G  932G  17% /gv/apps
*****demoninho*****
demoninho:/gv-data              2.5T  526G  2.0T  22% /gv/data
playerone:/gv-apps              1.1T  191G  932G  17% /gv/apps
*****zumbi*****
zumbi:/gv-data                  2.5T  526G  2.0T  22% /gv/data
playerone:/gv-apps              1.1T  191G  932G  17% /gv/apps
*****coragem*****
coragem:/gv-data                2.5T  526G  2.0T  22% /gv/data
playerone:/gv-apps              1.1T  191G  932G  17% /gv/apps
[root@playerone cfx]
\end{lstlisting}


O serviço do gluster entra automaticamente no religamento do sistema, entretanto não opera devido a ausência dos volumes. Uma vez que a montagem dos volumes esteja finalizada, funcionará normal e automaticamente. 

\section{Iniciar o \textit{slurm}}
% ---------------------------------------------------------------------------------------------------------
Este é o sistema que controla a fila de processos dos usuários a serem executados no cluster, o controlador de
recursos. Este sistema não é automaticamente estabelecido ao re-ligar o sistema e deve sê-lo manualmente.

São três serviços (\textit{daemons}):

\begin{enumerate}
\item \texttt{slurmctld}: Controlador propriamente dito. Recebe as informações dos nós e do mestre e controla
  a fila e o envio de \textit{jobs}.
  \item \texttt{slurmdbd}: Armazenador de dados. Os dados de utilização do cluster (especificamente, de uso do sistema de controle de fila) são coletados por esse serviço e armazenado em um banco de dados. Este banco de dados é provido e controlado por outro serviço.
  \item \texttt{slurmd}: O serviço local que controla os recursos de cada máquina. Há um versão deste serviço rodando em cada nó e uma no mestre, já que este também é um recurso do sistema.
\end{enumerate}

Notar que a ordem em que são iniciados os serviços não é só importante, mas obrigatória. Caso os serviços
sejam iniciados fora de ordem, o \textit{slurm} não funcionará corretamente.

Como iniciar o \textit{slurm}: %e verificar se foi iniciado corretamente:

\begin{enumerate}
\item iniciar no mestre o serviço de bando dados (\texttt{slurmdbd})
\item iniciar no mestre o serviço de controle (\texttt{slurmctld})
\item iniciar no mestre o serviço de local (\texttt{slurmd})
\item iniciar no nós o serviço de local (\texttt{slurmd})
\end{enumerate}

\begin{lstlisting}[language=bash,basicstyle=\small]
[root@playerone cfx] systemctl start slurmdbd
[root@playerone cfx] systemctl start slurmctld
[root@playerone cfx] systemctl start slurmd
[root@playerone cfx] bsh systemctl start slurmd
*****irineu*****
*****insolente*****
*****galo*****
*****marvin*****
*****mordac*****
*****demoninho*****
*****zumbi*****
*****coragem*****
[root@playerone cfx]
\end{lstlisting}

Verifique se o serviço subiu corretamente verificando a fila de \texttt{jobs}:

\begin{lstlisting}[language=bash,basicstyle=\small]
[root@playerone cfx] squeue 
             JOBID PARTITION     NAME     USER ST       TIME  NODES NODELIST(REASON)
[root@playerone cfx] 
\end{lstlisting}


% ---------------------------------------------------------------------------------------------------------
% Se quiser mostrar os status, voltar com isso pro lstlisting
% ---------------------------------------------------------------------------------------------------------
%[root@playerone cfx] systemctl status slurmdbd
%. slurmdbd.service - Slurm DBD accounting daemon
%   Loaded: loaded (/usr/lib/systemd/system/slurmdbd.service; disabled; vendor preset: disabled)
%   Active: active (running) since Thu 2019-10-03 10:14:26 -03; 5s ago
%  Process: 6897 ExecStart=/usr/sbin/slurmdbd $SLURMDBD_OPTIONS (code=exited, status=0/SUCCESS)
% Main PID: 6899 (slurmdbd)
%    Tasks: 5
%   CGroup: /system.slice/slurmdbd.service
%           |- 6899 /usr/sbin/slurmdbd
%
%Oct 03 10:14:26 playerone.usuarios.cdtn.br systemd[1]: Starting Slurm DBD accounting daemon...
%Oct 03 10:14:26 playerone.usuarios.cdtn.br systemd[1]: PID file /var/run/slurmdbd.pid not readable (yet?) after start.
%Oct 03 10:14:26 playerone.usuarios.cdtn.br systemd[1]: Started Slurm DBD accounting daemon.
% ---------------------------------------------------------------------------------------------------------


\section{Iniciar o \textit{Ganglia}}
% ---------------------------------------------------------------------------------------------------------
Este é o sistema que coleta os dados de utilização dos nós e do mestre e o apresenta em formato gráfico via página
web \url{https://playerone.usuarios.cdtn.br}.

O serviço de páginas do cluster é o Apache e o nome do serviço é \texttt{httpd}. Este serviço é iniciado automaticamente e
a página do LTHN estará no ar normalmente. Entretanto, não apresentará os dados de uso das máquinas. Isso é feito pelo
Ganglia, que não é re-estabelecido automaticamente.

O Ganglia é formado por dois serviços (\textit{daemons}) distintos:

\begin{enumerate}
\item \texttt{gmond}: o \texttt{daemon} de monitoramento propriamente dito. Coletor de dados local. Vai em cada nó e no mestre caso este também seja recurso, como no caso do \textit{cluster} LTHN.
\item \texttt{gmetad}: o serviço que coleta os dados do \texttt{gmond} e de outros \texttt{gmetad} caso exista mais de um e armazena os dados que serão apresentados na interface web.
\end{enumerate}

Os serviços são automaticamente iniciados na ligação do sistema. Entretanto, como isso é feito fora de ordem, o coletor de dados do nós não sabe e não é capaz de reconhecer quais os sistemas ativos. Portanto, devem ser reiniciados os
sistemas locais de monitoramento. Após o \textit{restart} destes serviços, deve ser reiniciado o serviço Apache.

Como iniciar o \textit{Ganglia}?

\begin{lstlisting}[language=bash,basicstyle=\small]
  [root@playerone cfx] systemctl restart gmond
  [root@playerone cfx] bsh systemctl restart gmond
  [root@playerone cfx] systemctl restart httpd
\end{lstlisting}

A partir de agora, será possível ver a utilização das máquinas do cluster pela interface web.

%Como verificar se o 
 %Cria uma seção
  \chapter{O \textit{gluster não funciona}}

O sistema \textit{gluster} precisa que todas as máquinas esteja sincronizadas. O sistema está configurado para buscar
a data atual em um local de rede. Caso o serviço não esteja funcionando ou as máquinas estejam com horários errados, o \textit{gluster} inicializará, mas ocorrerão erros espuŕios de difícil detecção.\\

Verificar a hora atual nas máquinas do cluster:

\begin{lstlisting}[language=bash,basicstyle=\small]
  [root@playerone ~] date
  Thu Oct  3 09:37:52 -03 2019
  [root@playerone ~] bsh date
  *****irineu*****
  Thu Oct  3 09:38:22 -03 2019
  *****insolente*****
  Thu Oct  3 09:38:23 -03 2019
  *****galo*****
  Thu Oct  3 09:38:23 -03 2019
  *****marvin*****
  Thu Oct  3 09:38:24 -03 2019
  *****mordac*****
  Thu Oct  3 09:38:24 -03 2019
  *****demoninho*****
  Thu Oct  3 09:38:24 -03 2019
  *****zumbi*****
  Thu Oct  3 09:38:25 -03 2019
  *****coragem*****
  Thu Oct  3 09:38:25 -03 2019
\end{lstlisting}

As diferenças em segundos estão corretas, já que o comando é executado sequencialmente.
No cluster LTHN, o serviço \texttt{chronyd} é quem controla o serviço \texttt{ntpd}. As configurações
e alterações devem ser feitas nele.

%\textbf{IMPORTANTE: Desligar o horário de verão que está configurado para o sistema} ao voltar de férias. Nota para o Vitor.

Devido as mudanças no firewall do CDTN, apenas a máquina \texttt{playerone}
acessa o servidor \texttt{ntpd [lagrange.cdtn.br]}.
Por essa razão, as máquinas escravas tiverem que ser reconfiguradas
para usar o mestre como provedor de horas.Essa manutenção emergencial ocorreu
apenas para as máquinas em uso nada data. São elas: \texttt{irineu, galo, marvin, zumbi coragem}.]

  O comando \texttt{chronyc sources} mostra quais servidores estão em uso pelo serviço de horas.
  
Com isso, caso as máquinas estejam desincronizadas, é necessário conferir se o servidor está com o horário atualizado:

%  [root@playerone ~]# chronyc sources
%  210 Number of sources = 2
%  MS Name/IP address         Stratum Poll Reach LastRx Last sample               
%  ===============================================================================
%  ^- lagrange.cdtn.br              2   8   377   124  -1376us[-1580us] +/-   46ms
%  ^* gps.jd.ntp.br                 1   7   377   123    -67us[ -271us] +/- 5463us


\begin{lstlisting}[language=bash,basicstyle=\small]
  [root@playerone ~]# timedatectl 
  Local time: Qua 2024-08-14 15:52:21 -03
  Universal time: Qua 2024-08-14 18:52:21 UTC
  RTC time: Qua 2024-08-14 18:52:21
  Time zone: America/Recife (-03, -0300)
  NTP enabled: yes
  NTP synchronized: yes
  RTC in local TZ: no
  DST active: n/a
\end{lstlisting}

Com o servidor mestre com o hora atualizada e o serviço disponível, basta forçar a atualização
da hora nas máquinas escravas:

\begin{lstlisting}[language=bash,basicstyle=\small]
  [root@playerone ~]# bsh.nomordac chronyc -a makestep
  *****irineu*****
  200 OK
  *****galo*****
  200 OK
  *****marvin*****
  200 OK
  *****zumbi*****
  200 OK
  *****coragem*****
  200 OK
\end{lstlisting}

É possível verificar que as máquinas agora utilizam o servidor \texttt{playerone} como provedor
de horas:

\begin{lstlisting}[language=bash,basicstyle=\small]
[root@playerone ~]# bsh.nomordac chronyc sources
*****irineu*****
210 Number of sources = 1
MS Name/IP address         Stratum Poll Reach LastRx Last sample               
===============================================================================
^* playerone.usuarios.cdtn.>     2   6   377    14  +5673ns[  +16us] +/- 5562us
*****galo*****
210 Number of sources = 1
MS Name/IP address         Stratum Poll Reach LastRx Last sample               
===============================================================================
^* playerone.usuarios.cdtn.>     2   6   377    15    +22us[  +29us] +/- 5570us
*****marvin*****
210 Number of sources = 1
MS Name/IP address         Stratum Poll Reach LastRx Last sample               
===============================================================================
^* playerone.usuarios.cdtn.>     2   6   377    11   -701ns[+6426ns] +/- 5563us
*****zumbi*****
210 Number of sources = 1
MS Name/IP address         Stratum Poll Reach LastRx Last sample               
===============================================================================
^* playerone.usuarios.cdtn.>     2   6   377    13  +1583ns[  +41us] +/- 5548us
*****coragem*****
210 Number of sources = 1
MS Name/IP address         Stratum Poll Reach LastRx Last sample               
===============================================================================
^* playerone.usuarios.cdtn.>     2   6   377    10    +65ns[  +20us] +/- 5569us
\end{lstlisting}


% Added some refernces used in different chapters
\bibliographystyle{plain}
\bibliography{main} % Referenciar os dois papers sobre o cluster (INAC 2017 e 2019)

\end{document}
