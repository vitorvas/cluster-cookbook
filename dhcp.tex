\chapter{A \textit{DHCP nas m�quinas n�s}}

A configura��o inicial do cluster fixava os IPs dos n�s de forma est�tica numa rede fict�cia (13.13.13.x). 
Isso inviabiliza o uso das m�quinas na rede CDTN. Para isso, � preciso reconfigurar as m�quinas de modo a que passem a usar um ip din�mico oferecido pela rede.

A primeira modifica��o � remover o arquivo \texttt{/etc/hosts} que cont�m a lista fixa de ips das m�quinas. Caso n�o deseje remover o arquivo, basta 
renome�-lo. Por exemplo:


\begin{lstlisting}[language=bash,basicstyle=\small]
  [root@playerone ~]# mv /etc/hosts /etc/hosts.backup_dia_mes 
\end{lstlisting}


Em sequida, � necess�rio adicionar um arquivo de configura��o para o protocolo DHCP. 
No diret�rio \texttt{/etc/NetworkManager/conf.d/}, criar o arquivo texttt{dhcp-client.conf} com o comando \texttt{touch} como no procedimento a seguir:

\begin{lstlisting}[language=bash,basicstyle=\small]
[root@coragem ~]# cd /etc/NetworkManager/conf.d/
[root@coragem ~]# touch dhcp-client.conf
[root@coragem conf.d]# ls
dhcp-client.conf
\end{lstlisting}

Neste arquivo deve ser adicionado o conte�do de configura��o exibido abaixo:

\begin{lstlisting}[language=bash,basicstyle=\small]
[main]
dhcp=internal
\end{lstlisting}

A partir desse instante, quando acionado, o servi�o DHCP saber� como configurar a interface.

%Entretanto, ainda s�o necess�rias algumas a��es. A primeira a��o consiste em desabilitar a interface de rede que est� configurada para a rede est�tica.
%\begin{lstlisting}[language=bash,basicstyle=\small]
%\end{lstlisting}

Se a interface a ser usada for, por exemplo, a interface \texttt{em2} (segunda porta da esquerda para a direita na parte traseiro do n�), deve-se configurar 
a interface:

\begin{lstlisting}[language=bash,basicstyle=\small]
nmcli connection modify em2 ipv4.dhcp-hostname coragem
nmcli connection modify em2 ipv4.dhcp-send-hostname yes
\end{lstlisting}

Logo a seguir coloc�-la ativa:

\begin{lstlisting}[language=bash,basicstyle=\small]
nmcli connection up em2
\end{lstlisting}

Vale ent�o re-iniciar o servi�o de rede: \texttt{systemctl restart NetworkManager}.

A partir de agora, a interface deve estar ativa. Pode-se verificar com o comando \texttt{nmcli}. Uma sa�da similar deve ser apresentada:

\begin{lstlisting}[language=bash,basicstyle=\small]
[root@coragem ~]# nmcli connection show
NAME  UUID                                  TYPE      DEVICE
em2   02f29314-b091-43dc-aee0-98d844e50626  ethernet  em2
em1   27f166ac-f6ba-4aee-9f62-ff3100e0492c  ethernet  --
em1   7b947e08-6d17-4208-8aa3-19f10096b3ce  ethernet  --
em4   d84e4915-9022-43e2-baf0-f4a65adb566f  ethernet  --
\end{lstlisting}

Agora o comando \texttt{nslookup} deve responder com o ip din�mico da m�quina quando acionado:

\begin{lstlisting}[language=bash,basicstyle=\small]
[root@coragem ~]# nslookup coragem
Server:         10.20.129.100
Address:        10.20.129.100#53

Name:   coragem.usuarios.cdtn.br
Address: 10.20.132.38

\end{lstlisting}


