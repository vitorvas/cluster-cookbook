\chapter{A \textit{rede não funciona}}

No CentOS 7 são basicamente dois sistemas de rede interligados: \textit{network.service} e \textit{Network.Manager}.
O primeiro é o sistema de rede propriamente dito e o segunto é o gerenciador gráfico, acessível via interface gráfica (na barra no canto superior direito).

Ironicamente, ambos estão interconectados. A tentativa de inabilitar (\texttt{systemctl disable Network.Manager}) pode deixar o cliente \textit{DHCP} órfão o que
impossibilitará a o re-início do serviço de rede (\texttt{systemctl restart network.service}).

%  [root@playerone ~]# chronyc sources
%  210 Number of sources = 2
%  MS Name/IP address         Stratum Poll Reach LastRx Last sample               
%  ===============================================================================
%  ^- lagrange.cdtn.br              2   8   377   124  -1376us[-1580us] +/-   46ms
%  ^* gps.jd.ntp.br                 1   7   377   123    -67us[ -271us] +/- 5463us

Lembrando que as interfaces \texttt{p1p1} e \texttt{p1p2} são as responsáveis pela conexão entre a \textit{Playerone} e as demais ``escravas''. Assim, ao menos
uma delas deve estar plugada (com o cabo de rede conectado) para que o \textit{Playerone} possa funcionar como gateway da sub-rede do cluster.

%\begin{lstlisting}[language=bash,basicstyle=\small]
%  [root@playerone ~]# timedatectl 
%  Local time: Qua 2024-08-14 15:52:21 -03
%  Universal time: Qua 2024-08-14 18:52:21 UTC
%  RTC time: Qua 2024-08-14 18:52:21
%  Time zone: America/Recife (-03, -0300)
%  NTP enabled: yes
%  NTP synchronized: yes
%  RTC in local TZ: no
%  DST active: n/a
%\end{lstlisting}
