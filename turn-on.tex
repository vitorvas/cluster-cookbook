\chapter{Ligando o sistema}
\label{chap:turn-on}

Em algumas ocasiões - falha no sistema de ar condicionado ou prolongada falta de energia não atendida pelo no-break - o \textit{cluster} deverá ser desligado ou desligará por si só. Como proceder:

\begin{enumerate}
    \item Religar todos os computadores e o NAS e aguardar que todos terminem o processo de boot. 
\end{enumerate}

Os volumes \textit{gluster} não serão montados automaticamente. Isso pode ser visto com o comando:

\begin{lstlisting}[language=bash,basicstyle=\small]
[root@playerone ~] df -h | grep gv
[root@playerone ~] 
\end{lstlisting}

\section{Montar volumes}
Nada retornado mostra que os volumes não estão montados. Para montar todos os volumes do sistema (notar o comando bsh montando para todos os nós:

\begin{lstlisting}[language=bash,basicstyle=\small]
[root@playerone cfx] mount -a  
[root@playerone cfx] df -h | grep gv
playerone:/gv-apps              1.1T  191G  932G  17% /gv/apps
playerone:/gv-data              2.5T  526G  2.0T  22% /gv/data
[root@playerone cfx] 
[root@playerone cfx] bsh mount -a
*****irineu*****
*****insolente*****
*****galo*****
*****marvin*****
*****mordac*****
*****demoninho*****
*****zumbi*****
*****coragem*****
[root@playerone cfx] bsh 'df -h | grep gv'
*****irineu*****
irineu:/gv-data                 2.5T  526G  2.0T  22% /gv/data
playerone:/gv-apps              1.1T  191G  932G  17% /gv/apps
*****insolente*****
playerone:/gv-apps              1.1T  191G  932G  17% /gv/apps
insolente:/gv-data              2.5T  526G  2.0T  22% /gv/data
*****galo*****
playerone:/gv-apps              1.1T  191G  932G  17% /gv/apps
galo:/gv-data                   2.5T  526G  2.0T  22% /gv/data
*****marvin*****
marvin:/gv-data                 2.5T  526G  2.0T  22% /gv/data
playerone:/gv-apps              1.1T  191G  932G  17% /gv/apps
*****mordac*****
mordac:/gv-data                 2.5T  526G  2.0T  22% /gv/data
playerone:/gv-apps              1.1T  191G  932G  17% /gv/apps
*****demoninho*****
demoninho:/gv-data              2.5T  526G  2.0T  22% /gv/data
playerone:/gv-apps              1.1T  191G  932G  17% /gv/apps
*****zumbi*****
zumbi:/gv-data                  2.5T  526G  2.0T  22% /gv/data
playerone:/gv-apps              1.1T  191G  932G  17% /gv/apps
*****coragem*****
coragem:/gv-data                2.5T  526G  2.0T  22% /gv/data
playerone:/gv-apps              1.1T  191G  932G  17% /gv/apps
[root@playerone cfx]
\end{lstlisting}


O serviço do gluster entrará automaticamente e, uma vez que a montagem dos volumes esteja finalizada, funcionará normalmente. 

\section{Iniciar o \textit{slurm}}

Este é o sistema que controla a fila de processos dos usuários a serem executados no cluster, o controlador de
recursos. Este sistema não é automaticamente estabelecido ao re-ligar o sistema e deve sê-lo manualmente.

São três serviços (\textit{daemons}):

\begin{enumerate}
\item \texttt{slurmctld.service}: Controlador propriamente dito. Recebe as informações dos nós e do mestre e controla
  a fila e o envio de \textit{jobs}.
  \item \texttt{slurmdbd.service}: Armazenador de dados. Os dados de utilização do cluster (especificamente, de uso do sistema de controle de fila) são coletados por esse serviço e armazenado em um banco de dados. Este banco de dados é provido e controlado por outro serviço.
  \item \texttt{slurmd.service}: O serviço local que controla os recursos de cada máquina. Há um versão deste serviço rodando em cada nó e uma no mestre, já que este também é um recurso do sistema.
\end{enumerate}

Notar que a ordem em que são iniciados os serviços não é só importante, mas obrigatória. Caso os serviços
sejam iniciados fora de ordem, o \textit{slurm} não funcionará corretamente.

Como iniciar o \textit{slurm}: %e verificar se foi iniciado corretamente:

\begin{enumerate}
\item iniciar no mestre o serviço de bando dados (\texttt{slurmdbd})
\item iniciar no mestre o serviço de controle (\texttt{slurmctld})
\item iniciar no mestre o serviço de local (\texttt{slurmd})
\item iniciar no nós o serviço de local (\texttt{slurmd})
\end{enumerate}

\begin{lstlisting}[language=bash,basicstyle=\small]
[root@playerone cfx] systemctl start slurmdbd
[root@playerone cfx] systemctl start slurmctld
[root@playerone cfx] systemctl start slurmd
[root@playerone cfx] bsh systemctl start slurmd
*****irineu*****
*****insolente*****
*****galo*****
*****marvin*****
*****mordac*****
*****demoninho*****
*****zumbi*****
*****coragem*****
[root@playerone cfx]
\end{lstlisting}

Verifique se o serviço subiu corretamente verificando a fila de \texttt{jobs}:

\begin{lstlisting}[language=bash,basicstyle=\small]
[root@playerone cfx] squeue 
             JOBID PARTITION     NAME     USER ST       TIME  NODES NODELIST(REASON)
[root@playerone cfx] 
\end{lstlisting}


% ---------------------------------------------------------------------------------------------------------
% Se quiser mostrar os status, voltar com isso pro lstlisting
% ---------------------------------------------------------------------------------------------------------
%[root@playerone cfx] systemctl status slurmdbd
%. slurmdbd.service - Slurm DBD accounting daemon
%   Loaded: loaded (/usr/lib/systemd/system/slurmdbd.service; disabled; vendor preset: disabled)
%   Active: active (running) since Thu 2019-10-03 10:14:26 -03; 5s ago
%  Process: 6897 ExecStart=/usr/sbin/slurmdbd $SLURMDBD_OPTIONS (code=exited, status=0/SUCCESS)
% Main PID: 6899 (slurmdbd)
%    Tasks: 5
%   CGroup: /system.slice/slurmdbd.service
%           |- 6899 /usr/sbin/slurmdbd
%
%Oct 03 10:14:26 playerone.usuarios.cdtn.br systemd[1]: Starting Slurm DBD accounting daemon...
%Oct 03 10:14:26 playerone.usuarios.cdtn.br systemd[1]: PID file /var/run/slurmdbd.pid not readable (yet?) after start.
%Oct 03 10:14:26 playerone.usuarios.cdtn.br systemd[1]: Started Slurm DBD accounting daemon.
% ---------------------------------------------------------------------------------------------------------

\section{Iniciar o \textit{Ganglia}}

Este é o sistema que coleta os dados de utilização dos nós e do mestre e o apresenta em formato gráfico via página
web \href{https://playerone.usuarios.cdtn.br}.

O serviço de páginas do cluster é o Apache e o nome do serviço é \texttt{httpd}. Este serviço é iniciado automaticamente e
a página do LTHN estará no ar normalmente. Entretanto, não apresentará os dados de uso das máquinas. Isso é feito pelo
Ganglia, que não é re-estabelecido automaticamente.

O Ganglia é formado por três serviços distintos:
%-------------------------------------------------------------------------------------------------------------
\chapter{O \textit{gluster não funciona}}

O sistema \textit{gluster} precisa que todas as máquinas esteja sincronizadas. O sistema está configurado para buscar
a data atual em um local de rede. Caso o serviço não esteja funcionando ou as máquinas estejam com horários errados, o \textit{gluster} inicializará, mas ocorrerão erros espuŕios de difícil detecção.\\

Verificar a hora atual nas máquinas do cluster:

\begin{lstlisting}[language=bash,basicstyle=\small]
  [root@playerone ~] date
  Thu Oct  3 09:37:52 -03 2019
  [root@playerone ~] bsh date
  *****irineu*****
  Thu Oct  3 09:38:22 -03 2019
  *****insolente*****
  Thu Oct  3 09:38:23 -03 2019
  *****galo*****
  Thu Oct  3 09:38:23 -03 2019
  *****marvin*****
  Thu Oct  3 09:38:24 -03 2019
  *****mordac*****
  Thu Oct  3 09:38:24 -03 2019
  *****demoninho*****
  Thu Oct  3 09:38:24 -03 2019
  *****zumbi*****
  Thu Oct  3 09:38:25 -03 2019
  *****coragem*****
  Thu Oct  3 09:38:25 -03 2019
\end{lstlisting}

As diferenças em segundos estão corretas, já que o comando é executado sequencialmente.
No cluster LTHN, o serviço \texttt{chronyd} é quem controla o serviço \texttt{ntpd}. As configurações
e alterações devem ser feitas nele.

\textbf{IMPORTANTE: Desligar o horário de verão que está configurado para o sistema} ao voltar de férias. Nota para o Vitor.
