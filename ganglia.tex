\chapter{A \textit{Adicionando um n� ao monitoramento Ganglia}}

O Ganglia funciona com dois servi�os:

\begin{enumerate}
gmetad
gmond
\end{enumerate}

Os dados s�o exibidos via interface web Apache e, como no nosso caso, o n� principal � monitorado mas tamb�m � o n� que fornece os dados � interface web do Ganglia, isso pode causar confus�o.

Ent�o, o que deve ser modificado?

\subsection{N�}

No n�, basta, no arquivo \texttt{/etc/ganglia/gmond.conf}, na se��o \texttt{upd_send_channel}, alterar 
a vari�vel \texttt{host} para o valor de ip din�mico do host playerone (10.20.132.100), como indicado abaixo:

\begin{lstlisting}[language=bash,basicstyle=\small]
udp_send_channel {
  #bind_hostname = yes # Highly recommended, soon to be default.
                       # This option tells gmond to use a source address
                       # that resolves to the machine's hostname.  Without
                       # this, the metrics may appear to come from any
                       # interface and the DNS names associated with
                       # those IPs will be used to create the RRDs.
#  mcast_join = 239.2.11.71
  host = 10.20.132.100 #13.13.13.1
  port = 8649
  ttl = 1
}\end{lstlisting}

A princ�pio � poss�vel usar o hostname (playerone) mas n�o foram feitos testes.

A seguir, reiniciar o servi�o \texttt{gmond.service} com \texttt{systemclt restart gmond.service}.

\subsection{Host}

No host, � o arquivo de meta dados que dever ser moficado.
Em \texttt{/etc/ganglia/gmetad.conf} alterar a linha com o \texttt{data_source} dos n�s monitorados:


\begin{lstlisting}[language=bash,basicstyle=\small]
data_source "cluster-LTHN" 10.20.132.100 10.20.132.157 10.20.132.38 <ip_do_no_a_adicionar>
\end{lstlisting}

Em seguida, reiniciar todos os servi�os.

\begin{lstlisting}[language=bash,basicstyle=\small]
 [root@playerone ganglia]# systemctl restart gmond.service gmetad.service httpd.service
\end{lstlisting}


Aten��o: ainda sem uma explica��o, �s vezes � necess�rio voltar ao n� e re-iniciar o servi�o \texttt{gmong.service} para 
que o n�mero de CPUs dispon�veis apresentado pela interface web do Ganglia reflita corretamente o n�mero de CPUs.

